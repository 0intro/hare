\chapter{Commercial Impact}
\section{Cray}
In August 2006, we had meetings with Cray to explore the use of Linux on their supercomputers 
in place of the Light Weight Kernel then shipping. The discussions aligned with the 
basic thesis of our Right Weight Kernel project, which was that a properly designed 
subset of a commercial kernel, coupled with a set of carefully designed extensions, could 
compete with Light Weight Kernels, in additon to bringing in the advantages of a more
standard programming environment. 

Larry Kaplan, of Cray, notes: ``The FastOS forums sponsored by the DOE
Office of Science were a very important part of Cray's investigations
into compute node operating systems for HPC.  The Right Weight Kernel
project in particular provided important information on the challenges
of using Linux which helped drive the research and development done at
Cray in order to deliver the Cray Linux Environment.  Cray fully
believes in the importance of the research community and the ability
for them to help influence our direction and success and looks forward
to future opportunities for collaboration.''

As part of Right Weight Kernel project, we created the Fixed Time
Quantum\cite{ftq} benchmark in order to support quantitative
evaluation of OS noise. Cray used this benchmark to evaluate and
improve the performance of their Compute Node Linux
product\cite{kaplan2007cray}.


\section{IBM}
IBM Research's involvement in this project has been extremely valuable in exploring
ways to broaden the applicability of the BlueGene hardware through alternative software
stacks.  It has allowed us to explore hardware features which aren't used by the production
software as well as explore alternative schemes for scalability and reliability which are
difficult or impossible under the production software.  This alternative view has given us
new insights which can then be incorporated into future hardware designs as well as future
production system software.

IBM also used FTQ to instrument the Blue Gene kernel\cite{bgpftq}.

\subsection{9p}

Our involvement in the HARE project has increased the visibility and use
of the Plan 9 resource sharing protocol, known as 9p, within the broader IBM community.
In particular, during the course of the HARE project we incorporated both 9p and the XCPU
workload management system into a hybrid computing platform called the Virtual Power
Environment, which was developed for the Marenustrum supercomputer at the Barcelona 
Supercomputing Center.  Infrastructures derrived from the original VPE idea are still
under development by HPC Links under the product named VERTEX~\cite{vertex}.
Experiences from the VPE effort were factored back into XCPU
resulting in the creation of the XCPU2~\cite{xcpu2} workload mangaement framework which then was
factored back into the HARE project as XCPU3~\cite{xcpu3} and the 
Unified Execution Model~\cite{uem}.  The fundamental components, Linux kernel support, and
advanced XCPU models were all developed as a direct result of the HARE funding.

Outside of the high performance computing area, the 9p protocol has achieved great success
within the emerging virtualization and cloud computing area as the defacto cloud computing
paravirtualized file system under Linux.  In coordination with IBM Research, development
teams have created a virtio~\cite{virtio} transport for 9p and incorporated a 9p server into
Qemu~\cite{qemu} which allows any KVM machines to be able to directly mount file systems
from the host~\cite{virtfs}.  This technology has been part of the main line kernel for about
a year and is being incorporated into the commercial Linux distributions.

\subsection{Hybrid Kernels}

Additionally, our experiences with hybrid environments on BlueGene with different kernel
types running on different nodes has led us towards a hybrid kernel model for future
massive multicore platforms where we are looking at taking advantage of running different
types of kernels on different cores in order to improve performance and efficiency by
insulating application cores from operating system noise.  These so called "hybrid" kernels
are currently being evaluated to be the default production kernel on future platforms.

\subsection{Analytics}

Several of the workload distribution methodologies we explored 
during the HARE project are also being evaluated for use in large-scale
analytics systems.  In particular, the unified execution model and PUSH~\cite{PUSH}
workflow are being looked at for application within analytics clusters as an alternative to
Hadoop for certain scenarios and the multipipe technology developed as part of HARE is
also being evaluated for integration into several internal projects.

In addition to workflow management, IBM is also exploring the application of
9p synthetic and static file system technologies to analytics file systems which
can be used with our execution model or with more conventional analytics software
such as Hadoop. 

\section{Coraid}
Coraid is a fast-growing vendor of network disk systems used in cloud computing centers. 
Coraid uses the trace software we developed\cite{plan9trace} for Blue Gene. 
This software extends both the kernel and the linker to enable convenient, efficient 
performance tracing of kernel functions. A measure of the efficiency of this software 
is that  all of Coraid's production
kernels ship with the tracing software always installed and enabled. 
The capability provided by the software is so useful, and the performance impact so small, 
that Coraid decided that it should always be available. 

\section{Google}
As a first step in the FAST-OS project, in 2005 we 
funded the creation of  a compiler toolchain (C compiler, linker, and assembler) for Plan 9 on the 
AMD/Intel 64-bit architectures, known as Opteron and EM64-T.
As per a basic requirement of the project, these
tools were created as open source. 

Google has recently released a new programming language called Go. Go supports, among other 
architectures, the AMD/Intel 64-bit CPUs. For these CPUs, Google adapted the  Plan 9 C toolchain
we created. There is hence a very direct link from the FAST-OS funding to Google's Go programming 
language, a language which is seeing very wide adoption. 
