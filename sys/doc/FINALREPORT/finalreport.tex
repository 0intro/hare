\documentclass{report}
\usepackage{graphicx}
\title{HARE: Final Report}
\author{Sandia National Labs, IBM, Bell Labs, Vita Nuova}
\date{\today}

\begin{document}
\maketitle
\tableofcontents
\pagebreak

This report documents the results of work done over a 6 year period under the FAST-OS 
programs. The first effort was called Right-Weight Kernels, (RWK) and was concerned with 
improving measurements of OS noise so it could be treated quantitatively; 
and evaluating the use of two operating systems, Linux
and Plan~9, on HPC systems and determining how these operating systems needed
to be extended or changed for HPC, while still retaining their general-purpose nature. 

The second program, HARE, explored the creation of alternative runtime models, 
building on RWK. All of the HARE work was done on Plan~9. The HARE reseachers
were mindful of the very good Linux and LWK work being done at other labs and saw no
need to recreate it. 

The organizations included LANL (RWK) and Sandia (RWK, HARE), as the PI moved to Sandia; 
IBM; Bell Labs; and Vita Nuova, as a subcontractor to Bell Labs. In any given year, 
the funding was suffcient to cover a PI from each organization part time. 

Even given this limited funding, the two efforts had outsized impact: 
\begin{itemize}
\item Helped Cray decide to use Linux, instead of a custom kernel, and provided
the tools needed to make Linux perform well
\item Created a successor operating system to Plan~9, NIX, which has been taken in 
by Bell Labs for further development
\item Created a standard system measurement tool, Fixed Time Quantum or FTQ, which is widely
used for measuring operating systems impact on applications
\item Spurred the use of the 9p protocol in several organizations, including IBM
\item Built software in use at many companies, including IBM, Cray, and Google
\item Spurred the creation of alternative runtimes for use on HPC systems
\item Demonstrated that, with proper modifications, a general purpose operating systems
can provide communications up to 3 times as effective as user-level libraries
\end{itemize}

We describe details of these impacts in the following sections. The rest of this 
report is organized as follows: first, we describe commercial impact; next, 
we describe the FTQ benchmark and its impact in more detail; operating systems
and runtime research follows; we discuss infrastructure software; and close with a 
description of the new NIX  operating system, future work, and conclusions. 
\chapter{Commercial Impact}
\section{Cray}
In August 2006, we had meetings with Cray to explore the use of Linux on their supercomputers 
in place of the Light Weight Kernel then shipping. The discussions aligned with the 
basic thesis of our Right Weight Kernel project, which was that a properly designed 
subset of a commercial kernel, coupled with a set of carefully designed extensions, could 
compete with Light Weight Kernels, in additon to bringing in the advantages of a more
standard programming environment. 

Larry Kaplan, of Cray, notes: ``The FastOS forums sponsored by the DOE
Office of Science were a very important part of Cray's investigations
into compute node operating systems for HPC.  The Right Weight Kernel
project in particular provided important information on the challenges
of using Linux which helped drive the research and development done at
Cray in order to deliver the Cray Linux Environment.  Cray fully
believes in the importance of the research community and the ability
for them to help influence our direction and success and looks forward
to future opportunities for collaboration.''

As part of Right Weight Kernel project, we created the Fixed Time
Quantum\cite{ftq} benchmark in order to support quantitative
evaluation of OS noise. Cray used this benchmark to evaluate and
improve the performance of their Compute Node Linux
product\cite{kaplan2007cray}.


\section{IBM}
IBM Research's involvement in this project has been extremely valuable in exploring
ways to broaden the applicability of the BlueGene hardware through alternative software
stacks.  It has allowed us to explore hardware features which aren't used by the production
software as well as explore alternative schemes for scalability and reliability which are
difficult or impossible under the production software.  This alternative view has given us
new insights which can then be incorporated into future hardware designs as well as future
production system software.

IBM also used FTQ to instrument the Blue Gene kernel\cite{bgpftq}.

\subsection{9p}

Our involvement in the HARE project has increased the visibility and use
of the Plan 9 resource sharing protocol, known as 9p, within the broader IBM community.
In particular, during the course of the HARE project we incorporated both 9p and the XCPU
workload management system into a hybrid computing platform called the Virtual Power
Environment, which was developed for the Marenustrum supercomputer at the Barcelona 
Supercomputing Center.  Infrastructures derrived from the original VPE idea are still
under development by HPC Links under the product named VERTEX~\cite{vertex}.
Experiences from the VPE effort were factored back into XCPU
resulting in the creation of the XCPU2~\cite{xcpu2} workload mangaement framework which then was
factored back into the HARE project as XCPU3~\cite{xcpu3} and the 
Unified Execution Model~\cite{uem}.  The fundamental components, Linux kernel support, and
advanced XCPU models were all developed as a direct result of the HARE funding.

Outside of the high performance computing area, the 9p protocol has achieved great success
within the emerging virtualization and cloud computing area as the defacto cloud computing
paravirtualized file system under Linux.  In coordination with IBM Research, development
teams have created a virtio~\cite{virtio} transport for 9p and incorporated a 9p server into
Qemu~\cite{qemu} which allows any KVM machines to be able to directly mount file systems
from the host~\cite{virtfs}.  This technology has been part of the main line kernel for about
a year and is being incorporated into the commercial Linux distributions.

\subsection{Hybrid Kernels}

Additionally, our experiences with hybrid environments on BlueGene with different kernel
types running on different nodes has led us towards a hybrid kernel model for future
massive multicore platforms where we are looking at taking advantage of running different
types of kernels on different cores in order to improve performance and efficiency by
insulating application cores from operating system noise.  These so called "hybrid" kernels
are currently being evaluated to be the default production kernel on future platforms.

\subsection{Analytics}

Several of the workload distribution methodologies we explored 
during the HARE project are also being evaluated for use in large-scale
analytics systems.  In particular, the unified execution model and PUSH~\cite{PUSH}
workflow are being looked at for application within analytics clusters as an alternative to
Hadoop for certain scenarios and the multipipe technology developed as part of HARE is
also being evaluated for integration into several internal projects.

In addition to workflow management, IBM is also exploring the application of
9p synthetic and static file system technologies to analytics file systems which
can be used with our execution model or with more conventional analytics software
such as Hadoop. 

\section{Coraid}
Coraid is a fast-growing vendor of network disk systems used in cloud computing centers. 
Coraid uses the trace software we developed\cite{plan9trace} for Blue Gene. 
This software extends both the kernel and the linker to enable convenient, efficient 
performance tracing of kernel functions. A measure of the efficiency of this software 
is that  all of Coraid's production
kernels ship with the tracing software always installed and enabled. 
The capability provided by the software is so useful, and the performance impact so small, 
that Coraid decided that it should always be available. 

\section{Google}
As a first step in the FAST-OS project, in 2005 we 
funded the creation of  a compiler toolchain (C compiler, linker, and assembler) for Plan 9 on the 
AMD/Intel 64-bit architectures, known as Opteron and EM64-T.
As per a basic requirement of the project, these
tools were created as open source. 

Google has recently released a new programming language called Go. Go supports, among other 
architectures, the AMD/Intel 64-bit CPUs. For these CPUs, Google adapted the  Plan 9 C toolchain
we created. There is hence a very direct link from the FAST-OS funding to Google's Go programming 
language, a language which is seeing very wide adoption. 

\chapter{Fixed Time Quantum}

Fixed Time Quantum\cite{ftq} is a tool for creating measurements of operating system noise, allowing users to do quantitative analysis 
with
standard tools such as Matlab and GNU Octave.

FTQ has found wide use in both the research and commercial world, having become one of the standard measurement tools for OS noise. 
Both IBM and Cray have used it to help tune their operating system kernels for minimal noise. 
FTQ results supported Cray's decision to move to Compute Node Linux, replacing the Catamount Light Weight Kernel originally delivered with those systems. Real 
application results supported the initial results from FTQ. 

The distinguishing feature of FTQ is that it measures work per fixed amount of time, rather than the more traditional time taken to do a fixed amount of work. The distinction is subtle but essential.  FTQ is written to carefully ensure that the time samples are for a fixed amount but also that the sampling interval is {\em stationary}. FTQ ensures this property by starting each sample interval at the correct point in team even if, due to interference, the previous sample interval took too long. For more discussion of how this is done the reader is referred to the original paper. 

As mentioned, FTQ allows users and vendors to use quantitative, rather than qualitative analysis.  As an example we will walk through real data taken on 
Blue Gene/P on a very early implementation of Plan 9. 
The FTQ program produces two output files: a so-called "counts" file, and a times file. These two files allow a user to reconstruct how much work was done per interval. The files 
are useful independent of each other: the counts data gives some idea of how much work was done, so
that the same binary, under different operating system, can compare how well the applications run.
The times data can be used to determine how well the algorithm that maintains stationary sampling is
doing its job; deviations of the time from a desired baseline can point to operating system, virtual
machine,  or even hardware problems: the times file was used to diagnose virtual machine interference on the Purple system at LLNL. Combined, the two files can be used for deeper analysis, combining both work and time information. 

While the raw data is useful, it is possible to misread too much into a raw data plot. In the original paper we showed two traces, one of them generated by white noise and the other by a very well-defined signal. To the naked eye, they are indistinguishable. Further processing of the data is essential to ensure that the measured data represents information, not white noise. 

In Figure \ref{rawdata}, we show a similar plot for three FTQ runs, on CNK, Linux, and Plan 9. What is of interest are not the actual numbers -- the Plan~9 binary 
is a different binary -- but the occurrence of spikes in the graph. One can gain some information immediately: there is clearly periodic interference of the same 
frequency for each operating system. Further investigation revealed that the 64-bit processor clock, comprised of two 32-bit counters, had a glitch when the
low-order 32-bit counter rolled over, leading to a slightly longer sample interval at that point. The algorithm quickly resynchronized, however, such that the 
counts over a long period were correct. This spike can easily be misinterpreted in the time domain graph; it does not have any impact on the frequency domain graph. 
\begin{figure}[h]
\begin{center}
 \includegraphics[width=4in]{raw.jpg}
 \caption{{\bf A plot of FTQ 'count' data for three Blue Gene/P operating systems, with sample
 number on the X axis and unit-less work on the Y axis.}}
\label{rawdata}
\end{center}
\end{figure}

Converting this data to the fequency domain is relatively straightforward. We present the octave script in Figure \ref{octave}. 
\begin{figure}[h]
\begin{center}
\begin{verbatim}
samples = readcounts('9ftq63_0_counts.dat');
samples = fitmax(samples);
[Psd,w] = pwelch(samples,[],[],[],810);
\end{verbatim}
\caption{Octave script for processing raw FTQ data. We only use the counts file in this case.}
\label{octave}
\end{center}
\end{figure}
The program reads the file in using a function that returns a one-dimensional vector. The function
{\tt fitmax} normalizes all the values in the vector to the maximum value. Finally, 
the code calls the octave pwelch function, which generates two vectors: a power spectrum estimation (Psd) and a set of frequencies(w). The parameters are the samples, and the 
cycle counter 
frequency (810 Mhz.). 
We show a plot of the Power spectral density for the three operating systems in Figure \ref{psd}
\begin{figure}[htbp]
\begin{center}
 \includegraphics[width=5in]{spectrum.jpg}
\caption{{\bf Power Spectral Density for the three operating systems, Frequency in HZ on the x axis
and unit-less amplitude on the Y axis.}}
\label{psd}
\end{center}
\end{figure}

The graph shows that CNK has a better noise figure than either of its two
more general purpose competitors. It is also possible to see
the frequencies at which noise spikes occur. In fact, an earlier version 
of this chart revealed a very distinct noise spike at 30 HZ. which we 
eliminated. Finally, it is easy to see that, while there is an apparent similarity in the time
domain between ZNK and ZeptoOS noise,they are no at all similar in the
frequency domain. One can not simply look at raw data graphs and 
reveal all the hidden information. 

FTQ allows us to quantitatively measure and analyze noise, using standard 
signal processing techniques. Spectral components of noise can be measured, 
traced back to their source, and eliminated. 


\chapter{Operating Systems Developments}
\section{Plan 9 Port to Blue Gene /L and /P}
\subsection{Large Pages}
\subsection{Shared Heap}
\section{New network communications models}
\subsection{Active Message Support}
\subsection{What are we calling Charles stuff?}

\chapter{Run Time Developments}
Developing runtimes for processes on Plan~9 was challenging, because we did not opt for the 
traditional HPC approach of placing device drivers in user level libraries. From the beginning of this 
project, we had decided to have the kernel control communications and  
resource management tasks that have 
been relegated to user level runtimes for decades. Returning control of these tasks to  the kernel
makes the HPC resource available to a much wider set of applications than just MPI programs. On 
many traditional HPC systems, high performance networks are not even visible to non-MPI 
applications. On Plan~9 on BG/P, even the command shell has access to the fastest networks: hence
it is 
possible to run a shell pipeline between compute nodes, without rewriting all the 
programs involved to use MPI. Parallel programs become more portable, and much easier to 
write and use. 

Removing device drivers from libraries removed all the attendant complexity from those libraries; 
many user-level libraries have more complex code than the Plan~9 kernel, and even the smallest 
of those libraries is larger than our kernel. Our limited-function MPI supports many MPI applications 
in less than 5000 lines.

User-level runtimes are not without their advantages. Most MPI libraries are thread-safe, and 
incorporate an activity scheduler that can choose activities with very low overhead. Because the 
device driver is contained in the process, pointers to data are valid for all the operations the runtime 
performs, including setting up DMA. There is no need to translate addresses, and 
on a simple system like Blue Gene, there is no need for complex page pinning software. 
These MPI runtimes in essence create a single-address-space operating system, complete with drivers 
and scheduling, that is "booted" by the host operating system and then left to run in control of the
node. Again, the advantages of the runtime include the implementation  of system call functionality 
with function calls; and a single
address space, with the attendant elimination of address translations and page faults and 
elimination of unnecessary copies. 

Leaving device drivers in the kernel requires reduced overhead in some areas. 
A system call takes 1-2 microseconds, on average; this time must be 
reduced when latency-sensitive operations take place. User-level 
addresses must be mapped into the kernel address space. In interrupt handlers, user-level 
memory is not easily accessed in Plan~9, since interrupt handlers have no user context. 

In this section we describe the research we have undertaken to address these needs. 

\section{Unified Execution Model}

In order to address the need for a more dynamic model of interacting with large scale machines,
we built the unified execution model (UEM)~\cite{uem} which combined interfaces for logical provisioning 
and distributed command execution with integrated mechanisms for establishing and maintaining 
communication, synchronization, and control.  The intent was to provide an environment which
would allow the end-user to treat the supercomputer as a seamless extension of his desktop
workstation -- making elements of his local environment and file system available on the 
supercomputer as well as providing interfaces for interactive control and management of 
processes running on top of the high-performance system.

Our initial attempt at providing the UEM infrastructure was based around an extension of 
some of our previous work in cluster workload management~\cite{xcpu}~\cite{xcpu2}.  This 
initial infrastructure, named XCPU$^3$ was built on top of the Inferno virtual machine which could be
hosted on a variety of platforms (Windows, Linux, Mac, Plan 9), while providing secure 
communication channels and dynamic name space management.  It functioned as a server
for the purposes of providing access to the elements of the end-users desktop as well
as an gateway to the interfaces for job control and management on the supercomputer.
The management interface was structured as a synthetic file system built into the 
Inferno virtual machine, which could be accessed by applications running on the host
either directly via a library or by mounting a 9P exported name space (using the Linux
v9fs 9P client support built into the Linux kernel).

A key aspect which allowed the UEM to scale was a hierarchical organization of node
aggregation which on bluegene matched the topology of the collective network.  It became
clear that at the core of the interaction of the UEM infrastructure were one to many
communication pipelines which were used to broadcast control messages and aggregate
response communication and reporting.  We extracted that key functionality into a new
system primitive called a multipipe~\cite{multipipe}.  The result dramatically simplified
the infrastructure and improved overall system performance.  It also became clear that
multipipes were a useful primitive for the construction of applications and other system
services.  We extended the design to incorperate many to many communication as well as
support for collective operations and barriers.

While Inferno provided a great environment on top of end-user desktops, we didn't want
to incur the overhead of running Inferno on all the compute nodes of the supercomputer.
Using multipipes as the core primative, we built a pair of synthetic file systems for 
Plan 9 which provided an interface for initiating execution on a node named execfs and
a second file system whose purpose was to coordinate groups of remote processes to allow
for fan-out computation and aggregate control named gangfs.  The gangfs file systems
were built to organize themselves hierarchically in a fashion similar to XCPU$^3$.  This
could then be mounted by front-end systems or end user workstations and interacted with
using Inferno as a gateway and server.

In order to make interactions with the UEM file system a bit more straightforward
we developed a new shell which incorporated support for multi-pipes and interactions
with the taskfs and gangfs interfaces.  
Push is a dataflow shell which uses two new pipe operators, fan out('|<') and 
fanin('>|') to create Directed Acyclic Graphs of processes.
This allows construction of complicated computational workflows 
in the same fashion one would create UNIX shell script pipelines.

\chapter{Infrastructure Developments}
\section{Blue Gene Debug File System}

The Blue Gene incorporates a management network with JTAG level access to every
core on the system which is uses for loading, monitoring and control.
Our Plan 9 infrastructure represents the external interface to this JTAG infrastructure as
a file system. This file system is served by a special application on an external system
that connects to a management port on the service node which accepts JTAG and system
management commands. Directory hierarchies are used to organize racks, midplanes,
node cards, nodes, and cores. Each core has its own subdirectory with a set of files rep­
resenting the core's state and control interfaces.

This has the effect of representing the entire machine\'s state as well as special files
for interacting with per-node consoles as a single file hierarchy with leaf nodes
reminicent of the the /proc file system from Linux (except more powerful).
The monitoring network and debug file system were vital contributors to the speed with
which we were able to bring up the system.  Since all nodes and state are represented, 
it eases the job of writing parallel debuggers and profilers.  Indeed, the Plan 9
debuggers are already written to work against such file systems and so this allowed us
to debug multi-node kernelcode as easily as we would have debugged a multi-process
applications.

\section{nompirun}

While the debug file system provided us a great environment for bring-up, it
did not play well with the existing management infrastructures in production 
environments.  To accomodate production environments we wrote management daemons
for our IO nodes which interacted with the production management infrastructure for
status reporting, parameter passing, and notification of termination of applications.
We also constructed a number of scripts for interacting with the cobalt job submission
system used at Argonne National Labs which automated launch of Plan 9 jobs including
configuration of back-links to the front-end file servers and reporting channels for
the aggreagted standard output and standard error of the jobs running on all the 
compute nodes.  This alternate infrastructure was critical to allowing us to bring up
applications with alternate (non-MPI-based) runtimes while still playing nicely with 
the production management software and job schedulers and was reused to enable the
Kittyhawk profiles. 

\section{Kittyhawk u-boot and cloud provisioning}

The production Blue Gene control system only
allows booting a single image on all nodes of a block. 
Initial firmware which initializes the hardware is loaded 
via the control network into each node of the machine.

For Kittyhawk profiles, we take over the node from the 
firmware via a generic boot loader which is placed on all 
nodes by the existing control system. The boot loader, U-Boot, 
offers rich functionality including a scripting language, 
network booting via NFS and TFTP, and a network console. We extended
U-Boot with the necessary device drivers for the Blue Gene
hardware and integrated the Lightweight IP stack [6, 7] to
provide support for TCP as a reliable network transport.

With this base functionality, individual nodes can be controlled 
remotely, boot sequences can be scripted, and specialized kernel 
images can be uploaded on a case-by-case, node-by-node basis. 
Users can then use simple command line tools to request a node
or a group of nodes from the free pool and push kernels to the
newly allocated group.  The kittyhawk infrastructure allows
users to construct powerful scripted environments and
bootstrap large numbers of nodes in a controlled, flexible,
and secure way.  Simple extensions to this model can also be used
to reclaim and restart nodes from failure or simply to reprovision
them with different base software.


\chapter{NIX}

NIX is an operating system designed for manycore CPUs in which not 
all cores are capable of running an operating system. Examples of such 
systems abound, most recently in the various GPU systems. While most 
heterogeneous systems treat the non-OS cores as a subordinate system, 
in NIX they are treated as peers (or even superiors) of the 
OS cores. NIX has been influenced by our work on Blue Gene and more traditional
clusters.  

NIX
features a heterogeneous CPU model and can use a shared address space
if it is available. 
NIX partitions cores by function: Timesharing Cores, or
TCs; Application Cores, or ACs; and Kernel Cores, or KCs.  One or more
TC runs traditional applications.  KCs are optional, running kernel
functions on demand.  ACs are also optional, devoted to running an
application with no interrupts; not even clock interrupts.  Unlike
traditional kernels, functions are not static: the number of TCs, KCs,
and ACs can change as needed.  Also unlike traditional
systems, applications can communicate by sending messages to the TC
kernel, instead of system call traps.  These messages are "active"
taking advantage of the shared-memory nature of manycore CPUs to pass
pointers to data and code to coordinate cores.

\chapter{Future Work}
%Over the period of these two projects, we have shown that with appropriate 
%modifications, general purpose operating systems can be used for HPC computers. 
%We have developed tools that show the quantitative benefit of these 
%This experience confirms the experience of industry. 
This research has suggested a number of new directions for HPC. 
\begin{itemize}
\item New operating systems that step outside the stale Light-Weight Kernel/General Kernel
argument. We have shown that one such kernel, NIX, can support the desirable attributes of both
and even outperform LWKs at the things they do best. 
\item Network IO can and should be done in the kernel, and future research should 
work out the API.
\item Network software that does not assume a perfect network needs to be created and put into use. 
\item File systems must change to make explicit use of hierarchy. The ratio of Compute Nodes to IO Nodes should not be 
statically defined by wires, as it is today, but defined by the needs of the application, and hence flexible. 
\end{itemize}

\chapter{Conclusions}

While no commercial HPC vendor is using Plan~9, they are using much of the other software we developed, 
including the 64-bit compiler toolchain, FTQ, and NIX. 
There are other lessons learned: 
\begin{itemize}
\item OS bypass is not required for good network performance. ``Conventional knowledge
bypass'' is much more important: if we can get runtime libraries to accomodate 
such concepts as a shared heap address space, we can reduce runtime and OS 
complexity and remove OS bypass for the most part. 
\item The idea of quantitative measurement of OS noise was controversial when we first proposed it in 2004. 
In fact, there is a wealth of signal processing software and knowledge that can be applied to HPC. We need
to move beyond qualitative descriptions to quantitative measurements. Adjectives should be avoided
when numbers can be supplied. 
\item Users want a Unix-like API, even on systems like Blue Gene which do not run Unix on all nodes. 
Key requirements such as sockets, dynamic libraries, and a reasonably standard file system
interface can not be waved away. Light Weight Kernels that fail to provide these services
will either change (as did the CNK) or be abandoned by the vendor (as was Catamount). 
\item As pointed out above, ``Thus, for reliable communication, the hardware level flow-control is neither necessary nor sufficient, 
and flow-control must be added somewhere in the multiplexing software stack.'' Communications networks
on future architectures could be made simpler, faster, more reliable, and lower power by taking 
this fact into account. 
\item We need to move beyond simply gluing hardware interfaces into existing kernel designs in non-optimal ways. 
One of the worst examples of this tendency can be seen in the Linux drivers for HPC networks: they all emulate an Ethernet. 
This emulation results in ineffeciencies and poor software structure. Ethernet interfaces are for Ethernet networks. There is no need to have a network interface for a
Torus emulate and Ethernet, as it does on Compute Node Linux on the Cray. 
\end{itemize}

This work was done as part of the FAST-OS program and its successor. 
The ground was prepared for the FAST-OS programs in a series of meetings held in
the eary 2000s. One of the PIs earliest talks dates from 2002. In 2003, we made
the following arguments: 
\begin{itemize}
\item Linux is fun and working now; enjoy it
\item Linux is not forever
\item Some fundamental properties of the Unix model have fatal flaws
\item We are entering a period of “architecture flux” 
\item We need diversity in OS research now to prepare for new, strange machines
\item DOE can succeed in creating a successful, diverse OS research community
\end{itemize}

We might ask: is any of this less true now that our community is faced with 
the challenge of scaling parallelism up 1000-fold? We would argue that 
the problem is even worse. While this program may have had successes, in some 
sense the community is 
still plowing the same ruts. The extant machines still largely run Linux, and the 
problems we predicted in 2003 (for the 2012 timeframe) are starting to 
come to pass: with each new version, Linux gets just a bit slower, as it grows in size 
and complexity and the processors do not get any faster. As we stated in 2003, 
``At some point, like all other OSes, it is [Linux] going to fall over''. 
Several large companies, in private conversation, have told the PI that one of the
biggest management headaches they have is dealing with the increasing complexity of 
the Linux kernel, version to version. 

Did FAST-OS succeed? That is a harder question. In the sense of industry impact, we 
clearly had great success. In the sense of building a vibrant community of OS researchers
in DOE, we had some success; there is very good OS work going on at Argonne, Oak Ridge, 
and Sandia, and on a smaller scale, other Labs. 

Those are hard-won gains, and maintaining them 
will not be easy: OS research continues to come under 
pressure for both budget reasons and 
from those who believe that industry will just supply an OS as a turnkey 
answer, in spite of all historical evidence to the contrary. No standard OS has ever worked
for the large scale without substantial measurement and change. 

DOE needs to build on the success of these programs by continuing
to fund new, innovative research that will 
help vendors set new directions. At the same time, DOE Labs should almost never be in the business of 
providing production software; nor should commercial uptake of every last bit of research be the measure 
by which the research is measured. Failure should always be an option. If there are no failures, 
then the research is not nearly aggressive enough. 

When FAST-OS started, the discussion was whether the DOE Labs would have any OS 
knowledge or capability left by the end of the decade. FAST-OS and its successor
program created a culture of competency in the DOE Labs and attracted some very 
capable commercial and research partners  from outside the labs. 
In the end, we count this research, 
and the larger program that funded it, as a success. Whether
the gains we have made will be maintained is a question that can only 
be answered at the end of {\em this} decade. 

%\bibliographystyle{abbrvnat} 
\bibliographystyle{plain}

\bibliography{all}

\appendix
\section{NIX}
See attached document.
This document will appear in Bell Labs Technical Journal and was the subject
of a talk at a Bell Labs Technical Conference in Antwerp, Belgium, Oct. 2011. 

\end{document}
