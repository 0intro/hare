\chapter{Infrastructure Developments}
\section{Blue Gene Debug File System}

The Blue Gene incorporates a management network with JTAG level access to every
core on the system which is uses for loading, monitoring and control.
Our Plan 9 infrastructure represents the external interface to this JTAG infrastructure as
a file system. This file system is served by a special application on an external system
that connects to a management port on the service node which accepts JTAG and system
management commands. Directory hierarchies are used to organize racks, midplanes,
node cards, nodes, and cores. Each core has its own subdirectory with a set of files rep­
resenting the core's state and control interfaces.

This has the effect of representing the entire machine\'s state as well as special files
for interacting with per-node consoles as a single file hierarchy with leaf nodes
reminicent of the the /proc file system from Linux (except more powerful).
The monitoring network and debug file system were vital contributors to the speed with
which we were able to bring up the system.  Since all nodes and state are represented, 
it eases the job of writing parallel debuggers and profilers.  Indeed, the Plan 9
debuggers are already written to work against such file systems and so this allowed us
to debug multi-node kernelcode as easily as we would have debugged a multi-process
applications.

\section{nompirun}

While the debug file system provided us a great environment for bring-up, it
did not play well with the existing management infrastructures in production 
environments.  To accomodate production environments we wrote management daemons
for our IO nodes which interacted with the production management infrastructure for
status reporting, parameter passing, and notification of termination of applications.
We also constructed a number of scripts for interacting with the cobalt job submission
system used at Argonne National Labs which automated launch of Plan 9 jobs including
configuration of back-links to the front-end file servers and reporting channels for
the aggreagted standard output and standard error of the jobs running on all the 
compute nodes.  This alternate infrastructure was critical to allowing us to bring up
applications with alternate (non-MPI-based) runtimes while still playing nicely with 
the production management software and job schedulers and was reused to enable the
Kittyhawk profiles. 

\section{Kittyhawk u-boot and cloud provisioning}

The production Blue Gene control system only
allows booting a single image on all nodes of a block. 
Initial firmware which initializes the hardware is loaded 
via the control network into each node of the machine.

For Kittyhawk profiles, we take over the node from the 
firmware via a generic boot loader which is placed on all 
nodes by the existing control system. The boot loader, U-Boot, 
offers rich functionality including a scripting language, 
network booting via NFS and TFTP, and a network console. We extended
U-Boot with the necessary device drivers for the Blue Gene
hardware and integrated the Lightweight IP stack [6, 7] to
provide support for TCP as a reliable network transport.

With this base functionality, individual nodes can be controlled 
remotely, boot sequences can be scripted, and specialized kernel 
images can be uploaded on a case-by-case, node-by-node basis. 
Users can then use simple command line tools to request a node
or a group of nodes from the free pool and push kernels to the
newly allocated group.  The kittyhawk infrastructure allows
users to construct powerful scripted environments and
bootstrap large numbers of nodes in a controlled, flexible,
and secure way.  Simple extensions to this model can also be used
to reclaim and restart nodes from failure or simply to reprovision
them with different base software.
