

%##################################################
%#
%# This is a simple template for a GTD project
%#
%#                        (c) John (EBo) David 2005
%#                             (ver) 0.1.1 - 050224
%#
%##################################################

\documentclass{article}
\usepackage{gtd}

%\GTDsilent     % silently strip all GTD Actions and stuff from document

%%% by default, citations in the document are off unless told
%%% otherwise.  The following two macro's overide this for the
%%% document
%\GTDcitation   % turn all citations on
%\GTDnocitation % turn all citations off

\usepackage[left=2cm,top=2cm,right=2cm,nohead,nofoot]{geometry}

\usepackage[colorlinks,plainpages=false]{hyperref}

\usepackage{amsmath}
\usepackage{natbib}

%\usepackage[nonumberlist,acronym]{glossaries}
%\newglossary[nlg]{symbols}{not}{ntn}{Symbols}
%\makeglossaries
%\loadglsentries[symbols]{\GTDUserHome/reference/symbols}
%\loadglsentries{\GTDUserHome/reference/acronyms}
%\loadglsentries{\GTDUserHome/reference/glossary}

\begin{document}
\GTDProject[proj=mpipefs]{mpipefs}
%  \title{TITLE} % the document title is pre-initialized with the
                 % project title above.  This can be overwridden
                 % using \title{}
%  \author{}     % document author
%  \date{}       % the date (default is set to today
\maketitle

\section*{Introduction}
This is an example project...

Brasil~\cite{Brasil:2011:BBR}

XCPU3~\cite{Pravin:2010:XWD}

\section*{Performance Testing}

To evaluate the overall performance of UEM's mpipe implementation,
gangfs's task execution time is broken into the following categories:
% Pavin 52

\begin{description}
  \item [Reservation:] Create a new session, and request the
    reservation by writing ``res n'' into the session's [ctl] file.
    Here n varies from 1 to 4098 representing the number of executions
    requested.
  \item [Execution:] Request the execution by writing exec date into
    the session's [ctl] file.
  \item [Input:] Distribute required input to all tasks by writing to
    the session's [stdin] file.
  \item [Aggregation:] Collect the output generated by all the
    executions by reading the session [stdio] file.
  \item [Termination:] Closing all the files and terminating the
    session.
  \item [Housekeeping:] Additional time taken before, between and
    after above steps.
\end{description}

\section*{Overhead Evaluation}

\subsection*{Initialization }
note: deployment without tasks

Housekeeping
Termination
Aggregation
Execution
Reservation

\subsection*{Execution Without Input}
note: deployment with tasks, but no input

Housekeeping
Termination
Aggregation
Execution
Reservation

\subsection*{Execution With Input}
note: deployment with tasks requiring input

Housekeeping
Termination
Aggregation
Input
Execution
Reservation


\subsection*{Cleanup}

\section*{Discussion}

\section*{Acknowledgments}

IBM, DOE, ...

\Action[]{look at tbonfs}
\Action[]{look at tree spawn or treefs?}
\Action[]{9p good for p2p but not task aggregation}

\Action[]{no select on Plan 9 is a problem because exportfs is acting
  as a proxy}

\Action[]{does uem have an equivelent to Brazil's ORS and IRS multiplexor?}

\Action[]{read thesis for details}

\GTDUseBib{paper.bib}
\GTDBib[plain]

%\glsaddall
%\printglossaries
%\printglossary[title={Symbols},type={symbols}]
%\printglossary[title={Acronyms},type={\acronymtype}]
%\printglossary[title={Glossary}]

\listActions
\listPurchases
\listContacts
\listProjects

\GTDoutput

\end{document}

Notes:

In the current version of UEM, the following sections inculde:

  Housekeeping
  Termination
  Aggregation
  Input
  Execution
  Reservation

