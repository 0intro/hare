\subsection{Examples}
While the PUSH shell handles much of the complexity of interacting with the
Brasil infrastructure, it is useful to see examples of how it interacts
with the underlying infrastructure in order to understand the various mechanisms
better.  These examples are given from the perspective of directly interacting
with the infrastructure file systems from a normal UNIX shell.

The first example presents how the default aggregation behaviour of Brasil can
used to deploy large number of applications.

\begin{verbatim}
$ less ./mpoint/csrv/local/clone
0 
\end{verbatim}
The above command is an example of creating a new session. The contents read
from the \texttt{[clone]} file represent the session-ID.  Now we use session 0
for performing actual execution.
\begin{verbatim}
$ echo "res 4" > ./mpoint/csrv/local/0/ctl
$ echo "exec date" > ./mpoint/csrv/local/0/ctl
$ cat ./mpoint/csrv/local/0/stdio
Fri May 7 13:53:58 CDT 2010
Fri May 7 13:53:58 CDT 2010
Fri May 7 13:53:58 CDT 2010
Fri May 7 13:53:58 CDT 2010
$
\end{verbatim}
The first echo command sends the request for reserving 4 remote resources. The
next echo command submits the request for executing the \texttt{date} command.
And the \texttt{cat} command on \texttt{[stdio]} returns the aggregated output
to the user.  This example shows all the complexities about finding, connecting
and using the remote resources is hidden behind the filesystem interface.
This approach can be used in the \textit{trivially parallelizable applications}
where the same application is deployed on all the nodes.

When constructing more complicated dataflow pipelines, Brasil handles
reserving the resources and setting up the pipe endpoints of the pipeline
components.
Instead of interacting with the aggregation points, dataflow applications (such
as the PUSH shell) can interact directly with the subsessions responsible for 
each pipeline component.

In this example, we will try to create a small pipeline of two commands
\texttt{date | wc}.  But we will create this pipeline across multiple nodes. 

Lets assume that session 0 is created by opening \texttt{[clone]} file as shown
in the previous example.  The following commands will create the desired
pipeline.

\begin{verbatim}
$ echo "res 2" > ./mpoint/csrv/local/0/ctl
$ echo "exec date" > ./mpoint/csrv/local/0/0/ctl
$ echo "exec wc" > ./mpoint/csrv/local/0/1/ctl
$ echo "xsplice 0 1" > ./mpoint/csrv/local/0/ctl
$ cat ./mpoint/csrv/local/0/1/stdio
1 6 29
$
\end{verbatim}

The first command \texttt{[exec date]} is sent to 0'th sub-session and the
second command \texttt{[exec wc]} is sent to the 1st sub-session.  The
\texttt{[xsplice 0 1]} request tells the parent session to redirect the output
of the 0'th session to the input of the 1st session.  The \textbf{xsplice}
command can be seen as a pipe operator of the shell script for redirecting the
output of one command to the input of other command.

The above example is equivalent of executing \texttt{date | wc} on the shell,
but with the difference that both commands are executed on a different remote
machines while sharing the same namespace.

