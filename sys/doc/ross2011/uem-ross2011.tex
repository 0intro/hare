% This is "sig-alternate.tex" V1.9 April 2009
% This file should be compiled with V2.4 of "sig-alternate.cls" April 2009
%
% This example file demonstrates the use of the 'sig-alternate.cls'
% V2.4 LaTeX2e document class file. It is for those submitting
% articles to ACM Conference Proceedings WHO DO NOT WISH TO
% STRICTLY ADHERE TO THE SIGS (PUBS-BOARD-ENDORSED) STYLE.
% The 'sig-alternate.cls' file will produce a similar-looking,
% albeit, 'tighter' paper resulting in, invariably, fewer pages.
%
% ----------------------------------------------------------------------------------------------------------------
% This .tex file (and associated .cls V2.4) produces:
%       1) The Permission Statement
%       2) The Conference (location) Info information
%       3) The Copyright Line with ACM data
%       4) NO page numbers
%
% as against the acm_proc_article-sp.cls file which
% DOES NOT produce 1) thru' 3) above.
%
% Using 'sig-alternate.cls' you have control, however, from within
% the source .tex file, over both the CopyrightYear
% (defaulted to 200X) and the ACM Copyright Data
% (defaulted to X-XXXXX-XX-X/XX/XX).
% e.g.
% \CopyrightYear{2007} will cause 2007 to appear in the copyright line.
% \crdata{0-12345-67-8/90/12} will cause 0-12345-67-8/90/12 to appear in the copyright line.
%
% ---------------------------------------------------------------------------------------------------------------
% This .tex source is an example which *does* use
% the .bib file (from which the .bbl file % is produced).
% REMEMBER HOWEVER: After having produced the .bbl file,
% and prior to final submission, you *NEED* to 'insert'
% your .bbl file into your source .tex file so as to provide
% ONE 'self-contained' source file.
%
% ================= IF YOU HAVE QUESTIONS =======================
% Questions regarding the SIGS styles, SIGS policies and
% procedures, Conferences etc. should be sent to
% Adrienne Griscti (griscti@acm.org)
%
% Technical questions _only_ to
% Gerald Murray (murray@hq.acm.org)
% ===============================================================
%
% For tracking purposes - this is V1.9 - April 2009

\documentclass{sig-alternate}

\begin{document}
%
% --- Author Metadata here ---
\conferenceinfo{ROSS}{'11 Tuscon, Arizona USA}
%\CopyrightYear{2007} % Allows default copyright year (20XX) to be over-ridden - IF NEED BE.
%\crdata{0-12345-67-8/90/01}  % Allows default copyright data (0-89791-88-6/97/05) to be over-ridden - IF NEED BE.
% --- End of Author Metadata ---

\title{Brasil}
\subtitle{Basic Resource Aggregation System Infrastructure Layer}
%
% You need the command \numberofauthors to handle the 'placement
% and alignment' of the authors beneath the title.
%
% For aesthetic reasons, we recommend 'three authors at a time'
% i.e. three 'name/affiliation blocks' be placed beneath the title.
%
% NOTE: You are NOT restricted in how many 'rows' of
% "name/affiliations" may appear. We just ask that you restrict
% the number of 'columns' to three.
%
% Because of the available 'opening page real-estate'
% we ask you to refrain from putting more than six authors
% (two rows with three columns) beneath the article title.
% More than six makes the first-page appear very cluttered indeed.
%
% Use the \alignauthor commands to handle the names
% and affiliations for an 'aesthetic maximum' of six authors.
% Add names, affiliations, addresses for
% the seventh etc. author(s) as the argument for the
% \additionalauthors command.
% These 'additional authors' will be output/set for you
% without further effort on your part as the last section in
% the body of your article BEFORE References or any Appendices.

\numberofauthors{3} %  in this sample file, there are a *total*
% of EIGHT authors. SIX appear on the 'first-page' (for formatting
% reasons) and the remaining two appear in the \additionalauthors section.
%
\author{
% You can go ahead and credit any number of authors here,
% e.g. one 'row of three' or two rows (consisting of one row of three
% and a second row of one, two or three).
%
% The command \alignauthor (no curly braces needed) should
% precede each author name, affiliation/snail-mail address and
% e-mail address. Additionally, tag each line of
% affiliation/address with \affaddr, and tag the
% e-mail address with \email.
%
% 1st. author
\alignauthor
Eric Van Hensbergen \\
       \affaddr{IBM Research}\\
       \email{bergevan@us.ibm.com}
% 2nd. author
\alignauthor
Pravin Shinde \\
       \affaddr{ETH Zurich}\\
       \email{shindep@student.ethz.ch}
\alignauthor
Noah Evans \\
       \affaddr{Alcatel-Lucent Bell Labs}\\
       \email{noah.evans@plan9.bell-labs.com}
}

\maketitle

\begin{abstract}
TBD: One paragraph summary, this paper can't fit any more than that.
\end{abstract}

\section{Introduction}

The deluge of huge data sets such as those provided by sensor
networks, online transactions, and petascale simulation provide 
exciting opportunities for data analytics.  
The scale of the data makes it increasingly difficult to process 
in a reasonable amount of time on isolated machines.
In the near future petascale and exascale simulation will make 
the involvement of secondary storage and disks impractical, driving
the need for infrastructures which provide in-situ analytics and
visualization capabilities~\cite{ma:in-site}.

Our experiences with existing tools for constructing dataflow workflows
for simulation, transformation, analysis and visualization were frustrating.
This was primarily due to the tight coupling of language, runtime, and
workflow tools which proved difficult to integrate with existing
applications.  
We also found many of these systems difficult (if not impossible)
to deploy on petascale clusters, particularly those with dynamic
resources such as clouds (where nodes might be added or removed based
on load, failure, or different system priorities).
% These might be other points to cover:
%%* tight coupling of language, runtime, and workflow methodology
%%* overly complex and difficult to deploy on petascale clusters or 
%%dynamic resources
%%* primarily targeted at post-processing
%This has lead to data flow systems emerging
%as the standard tool for solving research problems using these vast
%datasets.  In typical dataflow systems,
%runtimes~\cite{dean2008msd}~\cite{bialecki:hfr}~\cite{isard2007ddd}
%~\cite{streamit} and  define graphs of processes, the edges of the graphs
%representing pipes and their vertices representing computation on a
%system.  Within these runtimes a new class of
%languages~\cite{pike2005idp}~\cite{yu2008dsg}~\cite{olston2008pln} can
%be used by researchers to solve "pleasantly parallel" problems(problems where the individual elements of datasets are considered
%to be independent of any other element) more quickly without worrying
%about explicit concurrency.
%
%These languages provide automated control flow(typically matched
%to the architecture of the underlying runtime) and channels of
%communication between systems.  In existing systems, these workflows
%and the underlying computation are tightly linked, tying solutions
%to a particular runtime, workflow and language.  This creates
%difficulties for analytics researchers who wish to draw upon tools
%written in many different languages or runtimes which may be available
%on several different architectures or operating systems.

We observed that UNIX pipes were designed to get around many of these
incompatibilities, allowing developers to hook together tools written
in different languages and runtimes in ad-hoc fashions.  This allowed
tool developers to focus on doing one thing well, and enabled code
portability and reuse in ways not originally conceived by the tool
authors.  The UNIX shell incorporated a model for tersely composing
these smaller tools in pipelines (e.g. 'sort $|$ uniq -c'), creating
coherent workflow to solve more complicated problems quickly.  Tools
read from standard input and wrote to standard output, allowing
programs to work together in streams \emph{with no explicit knowledge
of this chaining built into the program itself}.

One to one pipelines such as those used by a typical UNIX shell,
can not be trivially mapped to streaming workflows which incorporate
one-to-many, many-to-many, and many-to-one data flows.  Additionally,
typical UNIX pipeline tools write data according to buffer boundaries
instead of record boundaries.  
%As \cite{pike2005idp} notes dataflow
%systems need to be able to cleanly separate input streams into
%records and then show that the order of these records is independent.
%By separating input and output into discrete unordered records data
%can be easily distributed and coalesced.
To address these issues we have implemented a prototype shell, which
we call PUSH, using dataflow principles and incorporating extended
pipeline operators to establish distributed workflows ---potentially
running on clusters of machines--- and correlate results. 
In order to scale this approach to large clusters of servers we 
implemented a workload distribution infrastructure which incorperated
dataflow communication constructs which we call \emph{XCPU3}.
%Our
%implementation is based on extending an existing shell, MASH\cite{mashman},
%from which we inherited a rich interpreted scripting language.  It
%treats variables as lists of strings and has no native handling for
%any other data type.  Integer expression handling and other facilities
%are provided by shell commands.  It has native regular expression
%support and it has a novel ability to do declarative shell programming
%through a make like syntax incorporated in the shell itself.
%
%We currently have a working prototype of the PUSH shell, which we
%can use to target local distributed clusters, dynamic clusters built
%using Amazon's EC2 cloud, and large scale clusters such as a Blue
%Gene running the kittyhawk infrastructure.  We are currently in the
%process of evaluating and optimizing performance for a variety of
%application types.
%
%the RC shell\cite{rcpaper} to easier integration into traditional unix
%systems like Linux. This version is simplified and closer to the
%bourne shell. The explicit goal of the new version of Push is
%integration with the Unified Execution Model(UEM)\cite{van-unified} which will
%allow the transparent distribution of processes and the connection
%of their communication channels between machines transparently.
%This work is taking place as part of the HARE project\cite{van2008holistic} 

The rest of this paper describes our design and implementation experience
for the XCPU3 infrastructure. The next section 
covers our core design principles behind the PUSH shell and how those are
reflected in the distribution and communication infrastructure.
Section 3 discusses our prototype implementation in more detail including
lessons learned while attempting to scale the prototype to thousands of cores.
Section 4 contains our evaluation of the overhead of the infrastructure 
when deployed on leadership class high-performance computing environments.
In Section 5 we will discuss potential improvements to address the overheads
measured results and explore opportunities for improvement in both the design 
and implementation of our approach.


\section{Desgin}
\emph{EVH: This section needs to be cut down a bunch -- perhaps by referencing XCPU2 and punting a lot of the file system decisions to it}

We decided to adopt a few guiding principles which will help us to reach our
goal of flexibility with scalability.  We also decided to keep the design simple,
and we believe that the simple design should lead to the simple and flexible
interface. We also hope that the lack of complexity may help in improving
scalability. In this section we present those decisions which influenced most
aspects of system design and implementation.

The key requirement for us was scalability to a large number of nodes.  We
planned to design the system without any central component which should have
knowledge of the entire system.  We plan to distribute and localize the
computations as well as decisions like scheduling, job management and
workload distribution/aggregation.

We avoid decision making based on global knowledge and promote use of local
information.  We hope that if each node tries to attempt a local optimum, we
will reach the global optimum.  This may not be true in all cases, We hope
that in those cases, we hope to perform acceptably well if not optimal.

As we plan to distribute and localize all functionalities, it was essential to
replicate these functionalities at multiple levels making localization
possible.  The granularity of functionality replication decides the granularity of control,
and hence influences the flexibility provided by the system.  As we aim for 
maximum flexibility, we have decided for replicating the following three
functionalities at each node.
\begin{enumerate}
\item \textbf{Resource reservation}: Each node should be able to reserve more
resources on it's own without involvement of any central entity.

\item \textbf{Job management}: Every node should be capable of starting and
managing new jobs using his reservations.

\item \textbf{Computation}: Every node should be able to perform the
computation by running the requested application in isolation and returning
the results.
\end{enumerate}

We intend to provide each node with the capability to perform all of these roles
simultaneously instead of binding them in one role at one time.  This design
makes the interface to every node a building block identical to each other, 
and provides the flexibility to build any structure with these building blocks.

There are a few downsides in making every node independent.  With independence
of every node, each node can be a source of failures and faults.
One will need to come up with better ways to deal with faults and
failures when so many sources of them are present.  This takes the XCPU3 in
realms of \textbf{Distributed Systems} opening many more possibilities and
questions.

For purpose of this exploration, we avoid these complexities by assuming a very
simple model for handling failures.  Any failure anywhere in the system will
result in the failure of the entire operation.  We assume that failures will
be in-frequent, so aborting and restarting operation should be acceptable for
such infrequent failures.

We want to keep XCPU3 interface agnostic from language, runtime and middleware.
Plan 9 has proven that the filesystem interface is very flexible and yet
powerful in the world of distributed applications.  We aim to follow the same
principle of \textbf{Everything is a file} from Plan 9.

Every node will provide access to its services via filesystem interfaces.
This interface will be exported as the filesystem over the network so that other
nodes can use it. Other nodes can mount this filesystem and use it as
interface for interacting with that node.  Multiple remote nodes can be
aggregated into the filesystem hierarchy providing a clean and easy way to
access them.

Multiple overlay views can be created by \textit{binding} the same filesystem
at multiple locations with different names.  This ability of creating ad-hoc
overlays allow users to arrange remote resources as per his needs without
worrying about their actual locations.

Other advantages of using filesystem interfaces are that 
\begin{enumerate}
\item Existing tools/commands used with traditional filesystem can be directly
used with XCPU3.

\item filesystems come with their own mechanism for access control list for
providing the security.  We can leave the security to these already proven 
mechanisms instead of implementing our own.

\item We inherit the ability to export, mount and bind the filesystem
without writing any explicit code for it.

\item Filesystem interfaces are simpler to program than socket interfaces.  This
can lead to simpler code and hence lesser bugs.

\item Users don't need special privileges or administrative access to interact
with the filesystem.  This simplifies the user experience in running XCPU3 based
applications.
\end{enumerate}

The filesystem interface has following limitations which affect our system
design.

\begin{enumerate}

\item The critical limitation concerns the POSIX standard for failure
reporting in filesystem.  POSIX standard dictates that file operations should
report their success or failure in the form of a single number which may not be
enough to provide the information about the reason behind failures.  Plan 9
overcomes this limitation by returning a string which can provide more
verbose information instead of a single number.  But this breaks the POSIX
compatibility.  So, we need to find an alternate way to report information
about failures in POSIX compatible way.

\item Another drawback is the way a failure of remote services is detected.
The filesystem interface relies on the underlying networking protocol for
detecting failures by waiting for timeouts, and then reporting them back to
users as an error.  This makes the filesystem interface less desirable where
quick failure detection and recovery is needed.
\end{enumerate}

Our decision to choose the filesystem interface despite of it's drawbacks is the
trade-off we are willing to do for flexibility and simplicity.  We limit
ourself with the assumption that failures will be infrequent.  With this
assumption, we are willing to accept the delays in reporting failures at the
remote end.

\subsection{Dataflow Shell Scripting}

\begin{figure}[htp]
\centering
\includegraphics[width=3in]{pipestruct.eps}
\caption{The structure of the PUSH shell}
\label{fig:pipestruct}
\end{figure}

We have added two additional pipeline operators to a traditional UNIX shell,
a multiplexing fan-out(\verb!|<![\emph{n}]), and a coalescing fan-in(\verb!>|!).
 
This combination allows PUSH to distribute I/O to and from multiple
simultaneous threads of control.
The fan-out argument \emph{n} specifies the desired degree of parallel
threading.  If no argument is specified, the default of spawning a new
thread per record (up to the limit of available cores) is used.  This can
also be overriden by command line options or environment variables.
The pipeline operators provide implicit grouping semantics allowing natural
nesting and composibility.
While their complimentary nature usually lead to symmetric
mappings (where the number of fan-outs equal the number of fan-ins), there is
nothing within our implementation which enforces it.
Normal redirections as well as application specific sources and sinks
can provide alternate data paths.
Remote thread distribution and interconnect are composed and managed
using synthetic file systems in much the same manner as Xcpu,\cite{xcpu}
pushing the distributed complexity into the middleware in an language and
runtime neutral fashion.

PUSH also differs from traditional shells by implementing native support for
record based input handling over pipelines. This facility is similar to the
argument field separators, IFS and OFS, in traditional shells which use a
pattern to determine how to tokenize arguments. PUSH provides two variables,
ORS and IRS, which point to record separator modules. These modules
(called multiplexors in PUSH) split data on record boundaries, emitting
individual records that the system distributes and coalesces.

The choice of which \emph{multipipe}, an ordered set of pipes, to target is
left as a decision to the module.
Different data formats may have different output requirements.Demultiplexing from a multipipe is performed by creating a many to one
communications channel within the shell. The shell creates a reader processes
which connects to each pipe in the multipipe. When the data reaches an
appropriate record boundary a buffer is passed from the reader to the shell
which then writes each record buffer to the output pipeline.

An example from our particular experience, Natural Language Processing, is
to apply an analyzer to a large set of files, a "corpus". User programs go
through each file which contain a list of sentences, one sentence per line.
They then tokenize the sentence into words, finding the part of speech and
morphology of the words that make up the sentence.
This sort of task maps very well to the DISC model. There are a large number of
discrete sets of data whose order is not necessarily important. We need to
perform a computationally intensive task on each of the sentences, which are
small, discrete records and ideal target for parallelization.

PUSH was designed to exploit this mapping. For example, to get a histogram of
the distribution of Japanese words from a set of documents using chasen,
a Japanese morphological analyzer, we take a set of files containing sentences
and then distribute them to a cluster of machines on our network. The command
is as follows:
\begin{verbatim}
push -c '{
  ORS=./blm.dis  du -an files |< xargs os \\
   chasen | awk '{print \$1}' | sort | uniq -c \\
   >| sort -rn
}'
\end{verbatim}

The first variable, ORS, declares our record multiplexor module, the intermediar
y
used to ensure that the input and output to distributed pipes are correctly
aligned to record boundaries. du -n gives a list of the files (note that our
du is a bit different from the canonical UNIX du, it replaces much of find's
functionality) which are then "fanned out"(\verb!|<!) using a combination
of a multipipes, and a \emph{multiplexor}
which determines which pipes are the targets of each unit of output.
This fanned out data goes to xargs on other machines which
then uses the filenames(sent from the instantiating machine) as arguments to
chasen. The du acts as a command driver, fanning out file names to the
individual worker machines. The workers then use the filenames input to
xargs, which uses the input filenames as arguments to xargs target command.
Using the output of the analyzer awk extracts the first line fields(Japanese
words) which are then sorted and counted using uniq.  Finally these word
counts are "fanned in"(\verb!>|!) to the originating machine which then
sorts them.


\subsection{Aggregation Infrastructure}

\subsection{Multi-Participant Pipelining}


\section{Implementation}

\subsection{Brasil}

\subsection{Central Services}

\subsection{Interface}

\subsection{Filters}


\subsection{Examples}
While the PUSH shell handles much of the complexity of interacting with the
Brasil infrastructure, it is useful to see examples of how it interacts
with the underlying infrastructure in order to understand the various mechanisms
better.  These examples are given from the perspective of directly interacting
with the infrastructure file systems from a normal UNIX shell.

The first example presents how the default aggregation behaviour of Brasil can
used to deploy large number of applications.

\begin{verbatim}
$ less ./mpoint/csrv/local/clone
0 
\end{verbatim}
The above command is an example of creating a new session. The contents read
from the \texttt{[clone]} file represent the session-ID.  Now we use session 0
for performing actual execution.
\begin{verbatim}
$ echo "res 4" > ./mpoint/csrv/local/0/ctl
$ echo "exec date" > ./mpoint/csrv/local/0/ctl
$ cat ./mpoint/csrv/local/0/stdio
Fri May 7 13:53:58 CDT 2010
Fri May 7 13:53:58 CDT 2010
Fri May 7 13:53:58 CDT 2010
Fri May 7 13:53:58 CDT 2010
$
\end{verbatim}
The first echo command sends the request for reserving 4 remote resources. The
next echo command submits the request for executing the \texttt{date} command.
And the \texttt{cat} command on \texttt{[stdio]} returns the aggregated output
to the user.  This example shows all the complexities about finding, connecting
and using the remote resources is hidden behind the filesystem interface.
This approach can be used in the \textit{trivially parallelizable applications}
where the same application is deployed on all the nodes.

When constructing more complicated dataflow pipelines, Brasil handles
reserving the resources and setting up the pipe endpoints of the pipeline
components.
Instead of interacting with the aggregation points, dataflow applications (such
as the PUSH shell) can interact directly with the subsessions responsible for 
each pipeline component.

In this example, we will try to create a small pipeline of two commands
\texttt{date | wc}.  But we will create this pipeline across multiple nodes. 

Lets assume that session 0 is created by opening \texttt{[clone]} file as shown
in the previous example.  The following commands will create the desired
pipeline.

\begin{verbatim}
$ echo "res 2" > ./mpoint/csrv/local/0/ctl
$ echo "exec date" > ./mpoint/csrv/local/0/0/ctl
$ echo "exec wc" > ./mpoint/csrv/local/0/1/ctl
$ echo "xsplice 0 1" > ./mpoint/csrv/local/0/ctl
$ cat ./mpoint/csrv/local/0/1/stdio
1 6 29
$
\end{verbatim}

The first command \texttt{[exec date]} is sent to 0'th sub-session and the
second command \texttt{[exec wc]} is sent to the 1st sub-session.  The
\texttt{[xsplice 0 1]} request tells the parent session to redirect the output
of the 0'th session to the input of the 1st session.  The \textbf{xsplice}
command can be seen as a pipe operator of the shell script for redirecting the
output of one command to the input of other command.

The above example is equivalent of executing \texttt{date | wc} on the shell,
but with the difference that both commands are executed on a different remote
machines while sharing the same namespace.


\section{Overhead Evaluation}

This section presents the evaluation of the XCPU3 infrastructure from the
perspective of quicker deployment of a large number of small jobs while giving
clean interfaces and abstractions.

We have performed our evaluations on a Blue Gene setup with 512 nodes.  This
setup is visually presented in a figure \ref{fig:hare}.  We run hosted Inferno
on all the compute nodes, IO nodes and the controller node.  The user interacts 
with the XCPU3 instance on the controller node for job submission.

\begin{figure}[h]
  \begin{center}
    \leavevmode
      \includegraphics[height=0.2\textheight,width=0.4\textwidth]
		{./img/EvaluationSetup}
    \caption{Setup for evaluation}
    \label{fig:hare}
  \end{center}
\end{figure}

\subsection{Execution}

Our first objective is to show how quickly we can do deployment and execution
of a large number of small applications. We have avoided using larger
applications as the bigger runtimes of larger applications tend to amortize the
overhead in the deployment of the application.  We have used the \texttt{date}
command as the application for deployment.  This is a small application and does
not need any external inputs and produces small output. Each deployment involves
session creation, reservation, execution, output aggregation and termination
of the session.  We deployed varying numbers of execution of this application on
the cluster of 512 nodes.  The number of requested executions increased
exponentially from 1 to 2048 executions.

\begin{figure}[h]
  \begin{center}
    \leavevmode
      \includegraphics[height=0.2\textheight,width=0.5\textwidth]
		{./img/linear}
    \caption{Comparison if XCPU3 with sequential deployment}
    \label{fig:sequential}
  \end{center}
\end{figure}

Figure \ref{fig:sequential} gives an initial perspective of how XCPU3 performs
relative to sequential performance.  This graph plots the total time taken by
XCPU3 and the hypothetical time it may take for performing the same amount of
work on one machine.  This graph shows that the XCPU3 is successfully able to
exploit the parallelism for deploying the jobs quickly. The XCPU3 deploys 2048
jobs in 12.66 seconds whereas sequential execution would take upto 510 seconds.
In the graph, the line showing sequential scaling looks exponential, but that is
because number of requested executions increase exponentially.

\subsection{Communication}

This section aims to further analyze the performance of XCPU3.  Again we are 
concentrating on similar deployment.  We have recorded the time taken by each
of the following stages in the deployment on the XCPU3 infrastructure.

\begin{enumerate}
\item Reservation: Create a new session, and request the reservation by writing
\texttt{res n} into the session \texttt{[ctl]} file.  Here \textbf{n} varies
from 1 to 2048 representing the number of executions requested. 

\item Execution: Request the execution by writing \texttt{exec date} into the
session \texttt{[ctl]} file.

\item Aggregation: Collect the output generated by all the executions by reading
the session \texttt{[stdio]} file.

\item Termination: Closing all the files and terminating the session.

\item Housekeeping: Additional time taken before, between and after above steps.
\end{enumerate}
Every deployment starts with the creation of the session followed by the
reservation, execution, aggregation and then ending with termination of the
session.  We have taken the average value over multiple runs for our analysis.

\begin{figure}[h]
  \begin{center}
    \leavevmode
      \includegraphics[height=0.2\textheight,width=0.5\textwidth]
		{./img/date_graph}
    \caption{Deployment without input}
    \label{fig:date_graph}
  \end{center}
\end{figure}

Figure \ref{fig:date_graph} presents the results of deployment of the date
command in the form of graph.  This graph presents the breakup time for various
stages of the deployment using the XCPU3 infrastructure.

From this graph we can observe that the session termination and the housekeeping
overheads are negligible compared to the time taken by reservation, execution
and aggregation.  So, we can ignore these two overheads in our future
evaluations.  For jobs of up to 128 deployments, the reservation time dominates
everything else.  But for larger numbers of deployment execution and
aggregation time increases rapidly while reservation time remains relatively
constant. This shows that reservation time is not directly dependent on
the number of deployments, whereas execution and aggregation time are directly
proportional to the number of deployments.

Now, let us try to analyze why reservation time is independent of the number of
deployments.  The reservation process involves traversing the underlying
topology tree of nodes till the reservation requirements are satisfied.  All the
children on the same level are traversed parallelly at the same time.  This way,
each level is traversed in the constant time, independent of number of nodes in
that level.  Another aspect of the reservation mechanism which helps here is
that the amount of data written and read from the \texttt{[ctl]} file and the
amount of data exchanged between nodes for communicating reservation request is
fixed in size and independent of the number of deployments requested.  With
these two properties, the reservation time becomes directly proportional to the
depth of the tree and not with the number of nodes.

We can observe the above relation in figure \ref{fig:date_graph}. The 
reservation time remains relatively constant for deployment requests from 1 to
8.  Then it sharply increases between 8 to 16 and remains almost constant for
all the requests between 16 to 2048.  This can be attributed to underlying
cluster topology. Figure \ref{fig:hare} shows the presence of the 8 IO nodes
in the first level. This enables satisfying the requests which are smaller
than 8 executions. For larger requests, one more level needs to be traversed
in the topology, introducing delays.  The time taken for reservation remains
almost constant between 16 and 2048 executions as all these reservation
requests essentially traverses the same depth.  We can conclude from these
observations that \textit{the time taken for the reservation is directly
proportional to the depth of the tree}.

Now let us discuss, why the same property is not exhibited by execution time or
aggregation time. We have discussed in the implementation chapter that all
read and write requests are performed in parallel between all the nodes in the
same level.  But the amount of data exchanged for aggregation and execution is
not constant.  This data is directly proportional to the number of nodes
involved.  With the increase in the number of requested deployments, the
amount of data to be exchanged also increases, leading to larger aggregation
time. The execution time is also similarly affected as all compute nodes will
try to fetch the binary of the executable from the initiating node leading to
the copy of the data. These observations lead us to to conclusion that
\textit{the time taken for the execution and aggregation is directly
proportional to the number of deployments requested}.

Our next evaluation involve the deployment of an executable \texttt{wc} which
needs input. This command counts the number of lines, words and characters in
the input file.  This is an interesting case for our infrastructure as this
deployment involves the distribution of inputs to all the sessions.  This
introduces a new stage in the deployment process in addition to the 5 stages we
described in the above section.  This stage will be the \textbf{input} stage and
involves distributing the input data to all the sessions which are responsible
for execution.  By default XCPU3 will broadcast the input to all the compute
nodes, but we have plans to introduce filters in the PUSH shell which can
partition the input given to the compute  nodes. 

Figure \ref{fig:wc_graph} presents the results of evaluations involving the
distribution of the input.


\begin{figure}[h]
  \begin{center}
    \leavevmode
      \includegraphics[height=0.2\textheight,width=0.5\textwidth]
		{./img/wc_graph}
    \caption{Deployment with input}
    \label{fig:wc_graph}
  \end{center}
\end{figure}

These results enforce our observations that reservation time is directly
proportional to the the depth of the tree whereas aggregation and execution time
are directly proportional to the number of deployments requested.  In addition
to these observations, we can also observe that the input time exhibits 
behavior similar to the execution and aggregation time.  This observation can be
attributed to the fact that input distribution implementation is similar
to the output aggregation implementation.  And also the amount of data to be
exchanged for input distribution depend on the number of deployments requested
as this data needs to reach all sessions responsible for execution.  This
increases the amount of data to be exchanged with any increase in the number of
the deployments requested.  We can conclude with this observation that
\textit{the input time is directly proportional to the number of deployments
requested}

\subsection{Dataflow}
The evaluations presented in this chapter helps us in understanding how fast
XCPU3 can deploy small jobs and aggregate the results produced by them.  Now
let us relate how these observations can justify our claims about dataflow
applications. Typical deployments of the dataflow applications are similar to
the above experiments as it involves starting up a large number of small jobs.
But the similarity is over at this point. Dataflow deployment does not always
involve the input distribution and output aggregation stages.  These deployments
work by feeding the output of one computation as input for other computation. 
At the end of the computation, a user needs to read the output of only selected
compute sessions which does not need any aggregation.  What we can conclude from
the above description is that XCPU3 performs really well in the stages like
reservation which are important for dataflow deployments.  The stages like input
distributions are output aggregation where XCPU3 is relatively slow are not
needed by dataflow deployments.  This makes XCPU3 ideal for rapid deployment of
dataflow workloads.

Unfortunately we do not have concrete measurements and evaluations to back our
predictions.  XCPU3 is one of the piece in the envisioned solution for problem
of efficient and easy deployment of large-scale dataflow applications.   We are
still working on the integration of userspace applications like
PUSH\cite{PODC:Push} with XCPU3 which will further simplify the dataflow
deployment.

XCPU3 is an infrastructure which provides the needed flexibility, speed and 
ease of use for dataflow workloads.  This chapter demonstrates the speed that
can be achieved.  It is difficult to measure the properties like \textit{ease of
use} and \textit{flexibility} but the examples presented in the filesystem
interface chapter should give some insights of all the possibilities opened up
by the XCPU3.


\section{Discussion}

\subsection{Weaknesses}
* Hierarchical mounts mean overhead
* aggreagtion models too simple
* fault tolerance sucks
* synch dataflow sucks
* async determinism difficult

\subsection{Missed Opportunities}
* missed opportunity: file system not incorporated
* need dynamic behavior within session

\subsection{Future Work} 


\section{Acknowledgements}
This work was supported in part by the Department of Energy
Office of Science under award number DE-FG02-08ER25851.

\bibliographystyle{plain}
\bibliography{references}

\end{document}
