% Thesis Abstract -----------------------------------------------------

%\begin{abstractslong}    %uncommenting this line, gives a different abstract heading
\begin{abstracts}        %this creates the heading for the abstract page

With growing acceptance of clusters and grids, the parallel programming
paradigm is coming out from the research community to the business domain.
Unfortunately the applications in the business domain are typically dataflow 
workloads which differ significantly from the traditional high
performance computing(HPC) workloads.  Most of the existing workload distribution
solutions concentrate more on HPC applications and are not efficient in quickly 
deploying the dynamic workload involving large number of small jobs which 
is the typical case in dataflow applications.
Also developing and deploying applications on existing infrastructures still 
is a non-trivial task requiring special runtimes, middleware support and/or language dependence.

We have explored an alternate approach for simplifying workload
distribution and aggregation based on synthetic filesystems and the private
namespace concepts of Plan 9.  Instead of sending workloads to a remote
cluster, XCPU3 works by bringing the remote cluster to the user's desktop,
leading to full control over the compute environment.   XCPU3 provides
a filesystem interface, making it runtime and language agnostic.  It also
allows executing multiple jobs simultaneously in isolation, leading to
better resource utilization.  XCPU3 makes all nodes independent and
equivalent in functionality leading to the ability to localize the
decision making and ability to handle dynamic workloads.  
XCPU3 provides easy to use, flexible and scalable infrastructure for 
workload distribution and aggregation that can be used by 
dynamic workloads and dataflow applications.

\end{abstracts}
%\end{abstractlongs}

% ----------------------------------------------------------------------

%%% Local Variables: 
%%% mode: latex
%%% TeX-master: "../thesis"
%%% End: 
